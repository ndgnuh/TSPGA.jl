\documentclass[aspectratio=169]{beamer}
\usepackage[utf8x]{vietnam}
\usepackage[vietnamese]{babel}
\usepackage{etoolbox}
\usepackage{tikz}
\usetikzlibrary{shapes.geometric}
\usepackage{booktabs}
\usepackage{subcaption}

\AtEndEnvironment{beamer@frameslide}{\vspace{-3em}}
\newcommand{\argmin}{\operatorname{argmin}}
\newcommand{\abs}[1]{\left\lvert#1\right\rvert}
\newcommand{\textabs}[1]{\left\lvert\text{#1}\right\rvert}
\newcommand{\Image}[1]{%
	\sbox0{\includegraphics[height=0.65\paperheight]{#1}}%
	\ifdim\wd0 < \textwidth
	\includegraphics[height=0.65\paperheight]{#1}%
	\else
	\includegraphics[width=\textwidth]{#1}%
	\fi%
}

\usetheme{hust}
\setbeamertemplate{caption}[numbered]

\title{GIẢI BÀI TOÁN NGƯỜI BÁN HÀNG}
\subtitle{SỬ DỤNG GIẢI THUẬT DI TRUYỀN}
\author{Nguyễn Đức Hùng--20173520}
\date{Tháng 7, 2021}
\titlegraphic{\includegraphics{sami-v2}}

\catcode`| = 13
\def|{\item}

\begin{document}
\begin{frame}[plain,noframenumbering]
	\maketitle
\end{frame}

\def\ft{Bài toán}
\begin{frame}{\ft}
	\begin{figure}
		\Image{traveling_salesman-Medium-930x620}
		\caption{Bài toán người đi du lịch (người bán hàng), \textit{hình ảnh chỉ mang tính chất minh hoạ}}
	\end{figure}
\end{frame}

\begin{frame}{\ft}{Mô hình hoá}
	\begin{itemize}
		| Thành phố $\to$ đỉnh
		| Đường đi $\to$ cạnh
		| Chi phí đi qua các thành phố  $\to$ trọng số trên cạnh
		| Tổng chi phí: 
			\begin{equation*}
			f(\text{đường đi}) = \sum {\text{Trọng số các cạnh trên đường đó}}
			\end{equation*}
		|[$\implies$] Cần tìm $\argmin{f}$
	\end{itemize}
\end{frame}

\def\ft{Giải thuật di truyền}

\begin{frame}
	{\ft}{Trạng thái}
	\begin{itemize}
			| Đồ thị $\to$ ma trận kề cận, e.g.:
				\begin{equation*}
					\mathbf{A} = \left[
						\begin{array}{cccccc}
						\infty & 12     & 29     & 22     & \cdots \\
						12     & \infty & 19     & 3      & \cdots \\
						29     & 19     & \infty & 21     & \cdots \\
						22     & 3      & 21     & \infty & \cdots \\
						\vdots & \vdots & \vdots & \vdots & \ddots \\
						\end{array}
					\right]_{n\times n}
				\end{equation*}
			| Trạng thái $\to$ mảng hoán vị, e.g.:
			\begin{equation*}
			\begin{tikzpicture}
				\draw 
				++(2em, 0) rectangle +(2em, 2em) node[midway] {$1$}
				++(2em, 0) rectangle +(2em, 2em) node[midway] {$2$}
				++(2em, 0) rectangle +(2em, 2em) node[midway] {$3$}
				++(2em, 0) rectangle +(2em, 2em) node[midway] {$\ldots$}
				++(2em, 0) rectangle +(2em, 2em) node[midway] {$n$}
				;
			\end{tikzpicture}
			\end{equation*}
			| Không gian trạng thái: tập tất hoán vị $n$ phần tử mà bắt đầu từ một vị trí nào đó.
	\end{itemize}
\end{frame}


\begin{frame}
	{\ft}{Tổng quan}
	\begin{figure}
		\Image{ga}
		\caption{Tổng quan các bước thuật toán}
	\end{figure}
	\vspace{-2em}
\end{frame}
\begin{frame}
	{\ft}{Các hàm lượng giá trị}
	\begin{itemize}
		| Hàm mục tiêu
			\begin{equation}
				f_{\mathbf{A}}(x)=\sum_{i=1}^{n-1}\mathbf{A}_{x[i],x[i+1]}+\mathbf{A}_{x[n],x[1]}
			\end{equation}
		| Hàm thích nghi:
			\begin{equation}
				g_{\mathbf{A}}(x)= \frac{1}{f_{\mathbf{A}}(x)}
			\end{equation}
		| Hàm chọn lọc: 
			\begin{equation}
				\Pr\left[x_{i}\text{ selected}\right]=\frac{g_{\mathbf{A}}\left(x_{i}\right)}{\sum_{k=1}^{n}g_{\mathbf{A}}\left(x_{k}\right)}
			\end{equation}
	\end{itemize}
\end{frame}

\begin{frame}
	{\ft}{Trao đổi chéo và đột biến}
	\begin{itemize}
			| Trao đổi chéo (1), với input: trạng thái $p_1, p_2$, đồ thị $\mathbf{A}$
			\begin{enumerate}
					| $W$ là trọng số các cạnh trong $p_1$
					| $k$ là số ngẫu nhiên, $k\sim\mathcal{U}_{\left[1,n-1\right]}$
					| Sắp xếp $p_1$ theo $W$ thu được $p'_1$
					| Lấy $k$ phần tử đầu của $p'_1$, ghép với $p_2 \setminus p'_1$ thu được con $c_1$
			\end{enumerate}
			| Trao đổi chéo (2): sử dụng $p_2$ như hoán vị của $p_1$, thu được con $c_2$.
		| Đột biến:
			\begin{equation*}
			\begin{tikzpicture}
				\draw 
				++(2em, 0) rectangle +(2em, 2em) node[midway] {$3$}
				++(2em, 0) rectangle +(2em, 2em) node[midway] {\textcolor{hustred}{$4$}}
				++(2em, 0) rectangle +(2em, 2em) node[midway] {$1$}
				++(2em, 0) rectangle +(2em, 2em) node[midway] {\textcolor{hustred}{$5$}}
				++(2em, 0) rectangle +(2em, 2em) node[midway] {$2$}
				;
			\end{tikzpicture}
				\raisebox{1ex}{${}\to{}$}
			\begin{tikzpicture}
				\draw 
				++(2em, 0) rectangle +(2em, 2em) node[midway] {$3$}
				++(2em, 0) rectangle +(2em, 2em) node[midway] {\textcolor{hustred}{$5$}}
				++(2em, 0) rectangle +(2em, 2em) node[midway] {$1$}
				++(2em, 0) rectangle +(2em, 2em) node[midway] {\textcolor{hustred}{$4$}}
				++(2em, 0) rectangle +(2em, 2em) node[midway] {$2$}
				;
			\end{tikzpicture}
			\end{equation*}
	\end{itemize}
\end{frame}

\begin{frame}
	{\ft}{Các tham số khác}
	\begin{table}
		\begin{tabular}{lr}
			\toprule
			\textbf{Tham số} & \textbf{Giá trị} 
			\\
			\midrule
			\midrule
			$\Pr[\text{Trao đổi chéo}]$ & $0.5$
			\\
			\midrule
			$\Pr[\text{Đột biến}]$ & $0.5$
			\\
			\midrule
			Kích cỡ quần thể & $\min(n, 100)$
			\\
			\midrule
			Số thế hệ không cải thiện trước khi dừng & $30$
			\\
			\midrule
			Sai lệch coi là không đáng kể & $10^{-32}$
			\\
			\midrule
			Số thế hệ tối đa & $2000$
			\\
			\bottomrule
		\end{tabular}
		\caption{Các tham số khác trong thuật toán}
	\end{table}
\end{frame}

\begin{frame}
	{\ft}{Độ phức tạp}
	\begin{itemize}
			| Tổng quan:
			\begin{equation}
				O\left(O\left(\text{chọc lọc}\right)\times\left(O\left(\text{trao đổi chéo}\right)+O\left(\text{đột biến}\right)\right)\right)
			\end{equation}
			| Chia nhỏ
			\begin{table}
				\begin{tabular}{lrrr}
					\toprule
					\textbf{Thao tác} & \textbf{Thời gian (b)} & \textbf{Thời gian (w)} & \textbf{Không gian}
					\\\midrule\midrule
					Trao đổi chéo (1) & $\Omega(n\log(n))$ & $O(n^2)$ & $O(n)$
					\\\midrule
					Trao đổi chéo (2) & $\Omega(n)$ & $O(n)$ & $O(n)$
					\\\midrule
					Đột biến & $\Omega(1)$ & $O(1)$ & $O(n)$
					\\\midrule        
					Lựa chọn & $\Omega(m)$ & $O(m)$ & $O(m)$
					\\
					\bottomrule
				\end{tabular}
				\caption{Độ phức tạp theo các quá trình, $n$ là kích thước bài toán, $m$ là kích thước quần thể}
			\end{table}
	\end{itemize}
	\vspace{-2em}
\end{frame}

\begin{frame}
	{\ft}{Độ phức tạp}
	\begin{figure}
		\begin{subfigure}{0.475\textwidth}
			\Image{benchmark-time.png}
			\caption{Thời gian (s)}
		\end{subfigure}
		\begin{subfigure}{0.475\textwidth}
			\Image{benchmark-memory.png}
			\caption{Không gian (MB)}
		\end{subfigure}
		\caption{Ước lượng độ phức tạp, kích thước bài toán $n \in \{25, 50, \ldots, 1000\}$.}
	\end{figure}
\end{frame}


\begin{frame}{\ft}
	\framesubtitle{Cải tiến}
	\begin{itemize}
		| Sử dụng yếu tố ngẫu nhiên \& ước lượng trực cảm chi phí trong trao đổi chéo
		| Sử dụng hai cách trao đổi chéo
		| Tỉ lệ trao đổi chéo và đột biết được thay đổi để phù hợp với hàm trao đổi chéo mới
		| Sử dụng hàm chọn lọc ngẫu nhiên
		| Thuật toán chạy nhanh và có tính ngẫu nhiên
			\begin{itemize}
				|[$\implies$] có thể chạy nhiều lần và lấy KQ tốt nhất
			\end{itemize}
	\end{itemize}
\end{frame}

\begin{frame}{Ví dụ}
	\begin{minipage}{0.5\textwidth}
		\begin{figure}
			\begin{tikzpicture}[
					every node/.style={draw=black,shape=circle,ultra thick}
				]
				\node (A) at (0, 4) {$1$};
				\node (B) at (-2,2) {$2$};
				\node (C) at (2, 2) {$3$};
				\node (D) at (0, 0) {$4$};
				\begin{scope}[
						every node/.style={draw=none, fill=white, shape=circle},
						every edge/.style={draw=red, ultra thick}
					]
					\path (A) edge node[midway]{$12$} (B);
					\path (A) edge node[midway]{$14$} (C);
					\path (A) edge node[near start]{$17$} (D);
					\path (B) edge node[near start]{$15$} (C);
					\path (B) edge node[midway]{$18$} (D);
					\path (C) edge node[midway]{$29$} (D);
				\end{scope}
			\end{tikzpicture}
			\caption{Đồ thị các thành phố và đường đi}
		\end{figure}
	\end{minipage}
	\begin{minipage}{0.475\textwidth}
		\begin{itemize}
				| Ma trận biểu diễn
				\begin{equation}
					\mathbf{A} = \left[
\begin{array}{cccc}
\infty & 12 & 14 & 17 \\
12 & \infty & 15 & 18 \\
14 & 15 & \infty & 29 \\
17 & 18 & 29 & \infty \\
\end{array}
\right]

				\end{equation}
				| Kết quả (với $\textabs{quần thể} = 5$)
				\begin{align*}
					TODO: KQ
				\end{align*}
		\end{itemize}
	\end{minipage}
\end{frame}

\begin{frame}[plain,noframenumbering]{}{}
	\textbf{\LARGE
		CẢM ƠN VÌ ĐÃ CHÚ Ý LẮNG NGHE.
	}
\end{frame}

\end{document}
